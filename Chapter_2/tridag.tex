\documentclass[12pt]{article}
	\addtolength{\oddsidemargin}{-.4in}
	\addtolength{\evensidemargin}{-.4in}
	\addtolength{\textwidth}{0.8in}
	\addtolength{\topmargin}{-0.8in}
	\addtolength{\textheight}{1.4in}
\usepackage{float}
\usepackage[margin=2.5cm]{geometry}
\usepackage{graphicx}
\usepackage[english]{babel}
%\usepackage{textgreek}
\usepackage{url}
\usepackage{amssymb}
\usepackage{epsfig}
\usepackage{bbm}
\usepackage{bbold}
\usepackage{graphicx}
\usepackage{amsmath}
\def\jddot#1{\stackrel{\bigdot\bigdot}{#1}}
\def\L{\mathcal L}
\def\e{\varepsilon}
\def\simlt{\stackrel{<}{{}_\sim}}
\def\simgt{\stackrel{>}{{}_\sim}}
\def\be{\begin{equation}}
\def\ee{\end{equation}}
\newcommand{\nucl}[3]{
\ensuremath{
\phantom{\ensuremath{^{#1}_{#2}}}
\llap{\ensuremath{^{#1}}}
\llap{\ensuremath{_{\rule{0pt}{.75em}#2}}}
\mbox{#3}
}
}





%\bibpunct{(}{)}{;}{a}{ }{,}
\begin{document} 
Let N be the number of grid points. 
\be
\psi_{N-1} -2\psi_N + \psi_{N+1}  = - \frac{\rho_N (\Delta x)^2}{\epsilon_0}
\ee 


The charge density at the last grid point ($\rho_N$) has a known value, made from a contribution of the plasma near the last grid point and particles that have deposited their charge on exiting the system. The grid point N+1 is not part of the system but it's potential
($\psi_{N+1}$)   is assumed to have the same value as the potential at the wall ($\psi_{N}$) for a perfect conductor. 
\be 
\psi_{N+1} = \psi_{N}
\ee 
Therefore 
\be
\psi_{N-1} -\psi_N   = - \frac{\rho_N (\Delta x)^2}{\epsilon_0}
\ee 

In solving the matrix equation 
\be
\begin{pmatrix}
  B_{1} & C_{1}  \\
  A_{2} & B_{2} & C_2 \\
        & A_3  & B_3 & C_3   \\
        & & \ddots & \ddots & \ddots \\
        & & &  A_N & B_N
\end{pmatrix}
\begin{pmatrix} 
 \psi_1  \\ 
 \psi_2  \\ 
 \psi_3  \\ 
 \vdots  \\
 \psi_N
\end{pmatrix}
= 
\begin{pmatrix} 
 \rho_1  \\ 
 \rho_2  \\ 
 \rho_3  \\ 
 \vdots  \\
 \rho_N
\end{pmatrix}
\ee         

$B_1=1 ,C1=0 , \rho_1=0$ to fix the potential on the left hand side to be 0V. And on the right hand side
$A_N =1, B_N = -1$ and $\rho_N$ is the charge deposited by particles in the last grid cell as well as the charge that has been deposited on the wall. 


\end{document}